\section{Conclusion}\label{sec:conclusion}
In this report, several methods from the course TTK4250 - Sensor Fusion, and the results from applying them to different data sets, have been stated and discussed. The methods implemented were the tracking method \acrshort{imm_pdaf}, the inertial navigation method \acrshort{eskf} and the \acrshort{slam} method \acrshort{ekf_slam}. These methods have been tuned to give \textit{good} results, as well as using consistency. By analysing the \acrshort{nis} and \acrshort{nees} where possible, it can be easier to see what changes could be made to improve the consistency. Using this, decent results were generated for all the assignments. 

The Kalman Filter, and the theory concerning it, are vital to all the methods discussed in this report. Especially interesting is the methods used for the \acrshort{eskf} and \acrshort{ekf_slam}, where the control input of a standard Kalman Filter has been substituted with a measurement. This allows for an easy way to incorporate multiple measurements into one filter, i.e. in the \acrshort{eskf} where the \acrshort{imu} measurements are used in the predict-step, while the \acrshort{gnss} measurements are used in the update-step. 

% This does not have to be long, but try to write a few reasonable closing remarks.

% Quickly say what we have implemented and what results we got.

% Say how looking at errors and especially NIS and NEES was extensively used in all parts - shows the greatness of consistency analysis

% write some inspiring shit about how all this is applications of the almighty kalman filter. Especially how the trick with letting control input be a measurement is used in both ESKF and EKFSLAM. 

