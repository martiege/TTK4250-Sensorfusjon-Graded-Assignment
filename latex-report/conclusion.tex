\section{Conclusion}\label{sec:conclusion}
In this report, several methods from the course TTK4250 - Sensor Fusion, and the results from applying them to different data sets, have been presented and discussed. The methods implemented were the tracking method \acrshort{imm_pdaf}, the inertial navigation method \acrshort{eskf} and the \acrshort{slam} method \acrshort{ekf_slam}. These methods have been tuned to give satisfactory results, in terms of estimation errors and consistency. By analysing the \acrshort{nis} and \acrshort{nees} where possible, it can be easier to see what changes could be made to improve the consistency. Using this, decent results were generated for all the assignments. 

The Kalman Filter, and the theory concerning it, are vital to all the methods discussed in this report. The report has presented three different applications based on the Kalman filter, showing its versatility and applicability to many different problems. But the implementations has also demonstrated some of the limitations of the Kalman filter. Most notably, poor scalability for systems with many states, as well as consistency issues for nonlinear models have been discussed.  

% This does not have to be long, but try to write a few reasonable closing remarks.

% Quickly say what we have implemented and what results we got.

% Say how looking at errors and especially NIS and NEES was extensively used in all parts - shows the greatness of consistency analysis

% write some inspiring shit about how all this is applications of the almighty kalman filter. Especially how the trick with letting control input be a measurement is used in both ESKF and EKFSLAM. 

