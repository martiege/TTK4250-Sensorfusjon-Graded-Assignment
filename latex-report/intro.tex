\section{Introduction}\label{sec:intro}
This report contains the results and discussions around the implementation and testing of the tracking method \acrfull{imm_pdaf}, the navigation method ESKF and the SLAM method \acrfull{ekf_slam}. The underlying techniques used were based on the theory of the Kalman Filter, though each algorithm have their own specialised implementation for their specific tasks. 

The first assignment is concerned with the tracking method \acrshort{imm_pdaf}, which is based on implementing multiple models for the same target, and then using the model which currently seems \textit{best} to describe the target, and use this information to track its movement. 

The second assignment is an implementation of the inertial avigation method \acrshort{eskf}. It predicts the state estimate of the vessel using \acrfull{imu} measurements, and then corrects the estimate using a \acrfull{gnss} measurement when available. This, together with a good bias model for the \acrshort{imu} measurements, can give very robust state estimates. 

Finally, the third assignment is an implementation of the \acrshort{slam} method \acrshort{ekf_slam}. This method is trying to both locate the robot and map the landmarks around it, by considering both \acrshort{gnss} measurements and odometry. As the robot's movement and the relative position of the landmarks would logically be very correlated, this can be used to estimate both location and mapping. 

All the assignments are organized such that the tuning and results are discussed separately for the different data sets, as well as some comparison between the data sets. There are also some comments and considerations of the methods implemented. Methods that were used to help the tuning process include \acrfull{rmse}, \acrfull{nis} and \acrfull{nees}, and were used where available.

% Overview and perspective

% Introduce the assignments

% Organization 



% Your introduction should contain an overview of the work you were assigned, as well as a few sentences putting the work into a larger perspective. You should also give a quick description of how the report is organized (as is done below).

% You should of course put most of the work into doing good work in the lab and then presenting it in the report. When presenting your work in the report, both content and presentation/layout matters. Since your only way of communicating your good effort in the lab is through writing about it here, the way you write about it is essential. This means that even if you have the very best controller but describe it poorly, you will probably not be rewarded for the good results. A plot showing perfect control is worth very little if it is not accompanied by a clear description of what it represents.

% Layout is naturally less important than content, but it still matters. You can think of report writing like selling an apartment; when you present your apartment for potential buyers you will of course clean the apartment and make it good looking. How clean the apartment is does of course not determine its value, but it is still important since it influences the subjective value your buyers will put on the apartment. 

% \subsection{Software}
% You are of course free to use whatever software you want for report writing. You can also submit a handwritten report, although this is probably not a great idea if your handwriting can be hard to read. 

% You can also use Word or a similar word processor. However, it is next to impossible to achieve decent layout with Word. The support for vector graphics (discussed later) is extremely poor, and text tends to look pretty bad (bad support for kerning and ligatures). Furthermore, math is both time consuming and difficult to input, and tends to look very ugly. In general, a report written in Word looks like a draft.

% It is strongly recommended to use Latex. Unless you tweak the layout too much, your report will almost certainly look very good. Although it can take a bit of effort to get started, it is also much quicker to use than Word and similar programs. The support for math and vector graphics is also great.

% If you are new to Latex, you can have a look at the source for this document to get started. You can also look at the presentation by~\cite{Berland2010} (in Norwegian) or consult~\cite{Oetiker2011}. Another good reason to learn Latex is that you probably don't want to write your master's thesis in something like Word, doing so would likely be very frustrating. Being reasonably fluent in Latex before you get that far will make your thesis work much smoother.

% Some of you are probably fluent in Latex and might plan to write the report using it. Please resist the temptation (if any) to change the fonts, make super fancy headers (they are not necessary for a report like this), change the margins, change the paragraph indentation and/or spacing, and similar things.

% A great tool for collaborating on Latex documents is ShareLaTeX at \url{https://www.sharelatex.com/}; if you use this you won't have to install anything on your computer. Texmaker at \url{http://www.xm1math.net/texmaker/} is a good cross-platform editor. Some people like Lyx, which is a Latex editor that behaves a little bit like Word. If you prefer to compile your Latex document on the command line, the latexmk \url{https://www.ctan.org/pkg/latexmk} command is a great tool included in most TeX distributions. There is also a simple Vim plugin that uses latexmk as its backend called LaTeX-BoX \url{https://github.com/LaTeX-Box-Team/LaTeX-Box}.

% \subsection{Other Comments}
% Unless you have a very good reason not to, you should write the report in English. If you have problems with Latex, the solution is usually just a few Google searches away.

% This report is organized as follows: \Cref{sec:prob_descr} contains some course specific equations, and some tips on how to create illustrations. Several \LaTeX{} tips can be found in \Cref{sec:latex_tips}, such as how to create a table and matrix equations. \Cref{sec:figures} contains some advice on using plots from MATLAB\@. The closing remarks are in~\Cref{sec:conclusion}, respectively. \Cref{sec:matlab} contains a MATLAB file while \Cref{sec:simulink} shows an example Simulink diagram. The Bibliography can be found at the end, on page~\pageref{sec:bibliography}.
