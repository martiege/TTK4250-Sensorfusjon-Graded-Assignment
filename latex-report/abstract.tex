\begin{abstract} 
This report will highlight and discuss the results of the three graded assignments for the course TTK4250 Sensor Fusion. Several useful and fascinating methods for combining and analysing sensor data have been discussed in this course, with special emphasis on three specific subjects: tracking, navigation and \acrfull{slam}. The code implemented was tested and tuned for both simulated and real datasets. 

% In the first assignment, an IMM-PDAF was implemented, in the second an ESKF was implemented, and in the third an EKFSLAM was implemented. 

% maybe less focus on the individual assignments, and more on the entirety of the course? 




%\addcontentsline{toc}{section}{Abstract} % add this if you want the abstract in the table of contents.
% This document outlines a few important aspects of a lab report. It contains some advice on both content and layout. The \LaTeX{} source for this document is also published, and you can use it as a template of sorts for your own report. You can find an up to date version of the source at \url{https://github.com/ntnu-itk/labreport}. The main file, ``labreport.tex'', defines the structure of the document. The ``preamble.tex'' file is the document preamble, and contains a lot of informative comments. The document is based on work done by Tor Aksel Heirung for TTK4135, and is now under continuous improvement by Andreas L. Fl{\aa}ten and Kristoffer Gryte (happily accepting suggestions and contributions from the community).

% When you write your own report, this section (the abstract) should contain a \emph{very} short summary of what the lab is about and what you have done.
\end{abstract}
